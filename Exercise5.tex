\documentclass[]{article}
\usepackage[T1]{fontenc}
\usepackage{lmodern}
\usepackage{amssymb,amsmath}
\usepackage{ifxetex,ifluatex}
\usepackage{fixltx2e} % provides \textsubscript
% use upquote if available, for straight quotes in verbatim environments
\IfFileExists{upquote.sty}{\usepackage{upquote}}{}
\ifnum 0\ifxetex 1\fi\ifluatex 1\fi=0 % if pdftex
  \usepackage[utf8]{inputenc}
\else % if luatex or xelatex
  \ifxetex
    \usepackage{mathspec}
    \usepackage{xltxtra,xunicode}
  \else
    \usepackage{fontspec}
  \fi
  \defaultfontfeatures{Mapping=tex-text,Scale=MatchLowercase}
  \newcommand{\euro}{€}
\fi
% use microtype if available
\IfFileExists{microtype.sty}{\usepackage{microtype}}{}
\usepackage[tmargin=1.5in]{geometry}
\ifxetex
  \usepackage[setpagesize=false, % page size defined by xetex
              unicode=false, % unicode breaks when used with xetex
              xetex]{hyperref}
\else
  \usepackage[unicode=true]{hyperref}
\fi
\hypersetup{breaklinks=true,
            bookmarks=true,
            pdfauthor={Riccardo De Bin and Vindi Jurinovic},
            pdftitle={Exercise 5},
            colorlinks=true,
            citecolor=blue,
            urlcolor=blue,
            linkcolor=magenta,
            pdfborder={0 0 0}}
\urlstyle{same}  % don't use monospace font for urls
\setlength{\parindent}{0pt}
\setlength{\parskip}{6pt plus 2pt minus 1pt}
\setlength{\emergencystretch}{3em}  % prevent overfull lines
\setcounter{secnumdepth}{5}

%%% Change title format to be more compact
\usepackage{titling}
\setlength{\droptitle}{-2em}
  \title{Exercise 5}
  \pretitle{\vspace{\droptitle}\centering\huge}
  \posttitle{\par}
  \author{Riccardo De Bin and Vindi Jurinovic}
  \preauthor{\centering\large\emph}
  \postauthor{\par}
  \predate{\centering\large\emph}
  \postdate{\par}
  \date{November 18, 2015}




\begin{document}

\maketitle


We have seen some statistical tools useful to compare groups in case of
discrete data. In particular, the first two questions refer to the use
of the chi-square and the Fisher exact tests.

\begin{enumerate}
\def\labelenumi{\arabic{enumi}.}
\itemsep1pt\parskip0pt\parsep0pt
\item
  Use a chi-square test in order to test whether the presence of chronic
  bronchitis and the current smoking status are independent.\\
\item
  Use a Fisher test to verify independence between sex and the presence
  of any liver disease.
\end{enumerate}

In the previous exercises, we have transformed the variable \texttt{hdl}
in order to fulfill the normality assumption necessary to perform the
t-test. Now, we know some non-parametric tests which allow us to test
out hypothesis without performing any transformation.

\begin{enumerate}
\def\labelenumi{\arabic{enumi}.}
\setcounter{enumi}{2}
\itemsep1pt\parskip0pt\parsep0pt
\item
  Perform a sign test both on \texttt{hdl} and on \texttt{logHdl} to
  test the hypothesis that the median of the cholesterol level is 1.30.
  Is the median significantly different from 1.30? Do you obtain the
  same results using \texttt{hdl} and \texttt{logHdl}? (remember to use
  the logarithm of 1.30 in the latter case).\\
\item
  Use a Mann-Whitney test to test the null hypothesis
  $H_0 : male\:weight = female\:weight$.
\end{enumerate}

Now let us go back to the NHANES data from your R Data Project. For
these new questions, it can be useful to draw a mosaic plot. In R, this
can be done through the function \texttt{mosaicplot()}. As usual, for
further information, it is possible to consult the related help page:
\texttt{\textgreater{} ?mosaicplot}.

\begin{enumerate}
\def\labelenumi{\arabic{enumi}.}
\setcounter{enumi}{4}
\itemsep1pt\parskip0pt\parsep0pt
\item
  Income and health.

  \begin{enumerate}
  \def\labelenumii{\alph{enumii})}
  \itemsep1pt\parskip0pt\parsep0pt
  \item
    It has been shown that there is a ``social gradient'' in health,
    such that the richer you are, the more likely you are to have better
    health. Plot general self-rated health against relative income so
    that you can get an impression whether this is confirmed by our
    data.\\
  \item
    Test the relation for statistical significance using an appropriate
    test.\\
  \end{enumerate}
\item
  Unemployment and depression

  \begin{enumerate}
  \def\labelenumii{\alph{enumii})}
  \itemsep1pt\parskip0pt\parsep0pt
  \item
    Recode the variable \texttt{jobstat} into a new dichotomous variable
    ``Unemployment yes/no''.\\
  \item
    Calculate a depression score by taking the mean across all items of
    the depression scale (``PHQ-9''). If the values for more than 3
    items are missing, assign the person a missing value. The resulting
    score may be treated as a metric variable.\\
  \item
    Investigate the hypothesis that unemployed people are more depressed
    than employed people. Check also whether the assumptions that your
    method of choice underlies are met.
  \end{enumerate}
\end{enumerate}

\end{document}
