\documentclass[]{article}
\usepackage[T1]{fontenc}
\usepackage{lmodern}
\usepackage{amssymb,amsmath}
\usepackage{ifxetex,ifluatex}
\usepackage{fixltx2e} % provides \textsubscript
% use upquote if available, for straight quotes in verbatim environments
\IfFileExists{upquote.sty}{\usepackage{upquote}}{}
\ifnum 0\ifxetex 1\fi\ifluatex 1\fi=0 % if pdftex
  \usepackage[utf8]{inputenc}
\else % if luatex or xelatex
  \ifxetex
    \usepackage{mathspec}
    \usepackage{xltxtra,xunicode}
  \else
    \usepackage{fontspec}
  \fi
  \defaultfontfeatures{Mapping=tex-text,Scale=MatchLowercase}
  \newcommand{\euro}{€}
\fi
% use microtype if available
\IfFileExists{microtype.sty}{\usepackage{microtype}}{}
\usepackage[margin=1.4in]{geometry}
\ifxetex
  \usepackage[setpagesize=false, % page size defined by xetex
              unicode=false, % unicode breaks when used with xetex
              xetex]{hyperref}
\else
  \usepackage[unicode=true]{hyperref}
\fi
\hypersetup{breaklinks=true,
            bookmarks=true,
            pdfauthor={Riccardo De Bin and Vindi Jurinovic},
            pdftitle={Exercise 2},
            colorlinks=true,
            citecolor=blue,
            urlcolor=blue,
            linkcolor=magenta,
            pdfborder={0 0 0}}
\urlstyle{same}  % don't use monospace font for urls
\setlength{\parindent}{0pt}
\setlength{\parskip}{6pt plus 2pt minus 1pt}
\setlength{\emergencystretch}{3em}  % prevent overfull lines
\setcounter{secnumdepth}{5}

%%% Change title format to be more compact
\usepackage{titling}
\setlength{\droptitle}{-2em}
  \title{Exercise 2}
  \pretitle{\vspace{\droptitle}\centering\huge}
  \posttitle{\par}
  \author{Riccardo De Bin and Vindi Jurinovic}
  \preauthor{\centering\large\emph}
  \postauthor{\par}
  \predate{\centering\large\emph}
  \postdate{\par}
  \date{October 26, 2015}




\begin{document}

\maketitle


~~~~~Writing your R-code, you can use the character \# to insert a
comment, so you can describe your commands (and remind yourself what you
did weeks or years ago). Example:

\begin{verbatim}
> x <- 2 + 4 + 1 # + 4 + 2
\end{verbatim}

Here, only the expression \texttt{x \textless{}- 2 + 4 + 1} will be
executed and \texttt{+ 4 + 2} will be ignored.\\

If you want to save a plot into a file named, say,
\texttt{'myPlot.pdf'}, use the following code:

\texttt{\textgreater{} pdf('myPlot.pdf')}\\This command specifies the
name and the file format. You can also choose
\texttt{jpeg(), png()},\ldots{}

\texttt{\textgreater{} plot(variable1, variable2, col=..., pch=..., ...)}\\Here
you are creating your plot with the chosen variables and arguments.

\texttt{\textgreater{} dev.off()}\\This command closes the graphics
device. Without it, the file is not created properly.\\\\

\section{R Project Part I}\label{r-project-part-i}

Here, we will answer some questions for your data project. The question
numbers are the same as on the question sheet.

\begin{quote}
2 Describe the US population with regards to:
\end{quote}

\begin{quote}
\begin{enumerate}
\def\labelenumi{\alph{enumi})}
\itemsep1pt\parskip0pt\parsep0pt
\item
  demographic characteristics (age, gender, ethnicity\ldots{}). Recode
  the age variable into following categories: 20-34, 35-49, 50-64,
  65-79, 80 or higher. Add this new variable (a factor!) to your data
  set.
\item
  self-rated health.
\end{enumerate}
\end{quote}

\begin{quote}
3 Lifetime prevalence of cancer in the population
\end{quote}

\begin{quote}
\begin{enumerate}
\def\labelenumi{\alph{enumi})}
\itemsep1pt\parskip0pt\parsep0pt
\item
  Estimate the lifetime prevalence of cancer. Can you also give an
  interval estimate?
\item
  What are the prevalences estimates in those who were exposed to
  pollutants at work for a longer time period, and in those who weren't?
  Is there a significant difference in prevalence between these two
  subgroups?
\end{enumerate}
\end{quote}

\begin{quote}
4 HDL cholesterol and gender
\end{quote}

\begin{quote}
\begin{enumerate}
\def\labelenumi{\alph{enumi})}
\itemsep1pt\parskip0pt\parsep0pt
\item
  Look at the distribution of high-density lipoprotein (HDL) cholesterol
  levels. What shape does it have? Apply an appropriate transformation
  to normalize HDL and save it as a new variable. (We already did this
  last week.)
\item
  Is there a significant difference between men and women in HDL
  cholesterol levels (using normalized variable)?
\end{enumerate}
\end{quote}

\end{document}
