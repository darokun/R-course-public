\documentclass[]{article}
\usepackage[T1]{fontenc}
\usepackage{lmodern}
\usepackage{amssymb,amsmath}
\usepackage{ifxetex,ifluatex}
\usepackage{fixltx2e} % provides \textsubscript
% use upquote if available, for straight quotes in verbatim environments
\IfFileExists{upquote.sty}{\usepackage{upquote}}{}
\ifnum 0\ifxetex 1\fi\ifluatex 1\fi=0 % if pdftex
  \usepackage[utf8]{inputenc}
\else % if luatex or xelatex
  \ifxetex
    \usepackage{mathspec}
    \usepackage{xltxtra,xunicode}
  \else
    \usepackage{fontspec}
  \fi
  \defaultfontfeatures{Mapping=tex-text,Scale=MatchLowercase}
  \newcommand{\euro}{€}
\fi
% use microtype if available
\IfFileExists{microtype.sty}{\usepackage{microtype}}{}
\usepackage[tmargin=1.5in]{geometry}
\ifxetex
  \usepackage[setpagesize=false, % page size defined by xetex
              unicode=false, % unicode breaks when used with xetex
              xetex]{hyperref}
\else
  \usepackage[unicode=true]{hyperref}
\fi
\hypersetup{breaklinks=true,
            bookmarks=true,
            pdfauthor={Riccardo De Bin and Vindi Jurinovic},
            pdftitle={Exercise 1},
            colorlinks=true,
            citecolor=blue,
            urlcolor=blue,
            linkcolor=magenta,
            pdfborder={0 0 0}}
\urlstyle{same}  % don't use monospace font for urls
\setlength{\parindent}{0pt}
\setlength{\parskip}{6pt plus 2pt minus 1pt}
\setlength{\emergencystretch}{3em}  % prevent overfull lines
\setcounter{secnumdepth}{5}

%%% Change title format to be more compact
\usepackage{titling}
\setlength{\droptitle}{-2em}
  \title{Exercise 1}
  \pretitle{\vspace{\droptitle}\centering\huge}
  \posttitle{\par}
  \author{Riccardo De Bin and Vindi Jurinovic}
  \preauthor{\centering\large\emph}
  \postauthor{\par}
  \date{}
  \predate{}\postdate{}


\usepackage{wasysym}


\begin{document}

\maketitle


~~~~~Today we will start to analyze the NHANES dataset from your R Data
Project. Choose one sub-sample to work with and think of a name for your
data set (say, \texttt{nhanes}, or \texttt{dataproject}, or just
\texttt{tab} for table, or any other name you like\ldots{}). Load the
data into your R workspace with the command:

\begin{verbatim}
> name_of_your_data <- read.table('name of the file', sep='/t', header=TRUE)
\end{verbatim}

Here, \texttt{sep} is the field separator character. Values on each line
of the file are separated by this character (you can check this by
opening the file in a text editor, or you can just try out some
separators until you get the right one). \texttt{header=TRUE} means that
R should interpret the first row of the data set as the variable
names.\\

Try to answer the questions by yourself. If you don't know the name of
some function, say sequence, try \texttt{??sequence}. To find out more
about a function and its arguments, use \texttt{?function\_name}.\\

\section{Getting familiar with the data
set}\label{getting-familiar-with-the-data-set}

\begin{itemize}
\itemsep1pt\parskip0pt\parsep0pt
\item
  What is the dimension of the data set? How many rows (samples), and
  how many columns (variables) does the data set contain? What are the
  variable names of the data set?\\
\item
  All the variables in the data set are either of a class
  \texttt{integer}, \texttt{numeric} (i.e., they are all interpreted as
  numbers by R) or \texttt{boolean} (i.e., logical). However, some of
  the variables should be factors rather than numerical variables.
  Change the class of these variables with the function
  \texttt{as.factor}. Save the new data set as an \texttt{.Rdata} file.
  Attach the data frame so you can assess the variable names directly.\\
\item
  How many women and how many men are there in your data set?\\
\item
  What is the mean BMI in the overall population? What is the mean BMI
  for men and women?\\
\item
  Who has a higher mercury level in blood: men or women? People with
  chronic bronchitis or people without it? `Hispanic', `White', `Black'
  or `Other/Mixed' people?\\
\item
  Use the function \texttt{summary} to get summarized information on all
  the variables in the data set.\\
\end{itemize}

\section{Plots}\label{plots}

\begin{itemize}
\itemsep1pt\parskip0pt\parsep0pt
\item
  Plot the variable \texttt{rr\_sys} as a function of \texttt{bmi}. Try
  out different types o point characters (function argument
  \texttt{pch}) and colors (argument \texttt{col}) and choose the
  prettiest ones $\smiley$. Label the x- and y-axis in the plot with the
  corresponding variable names (function arguments \texttt{xlab} and
  \texttt{ylab}). Think of a suitable title and add it to your plot.
  Type \texttt{?plot} for more information on the function and its
  arguments and try out some of these.\\
\item
  Now we want to plot the variable \texttt{rr\_sys} against
  \texttt{diab\_lft}. Which plot should we use here? Create the plot and
  choose different colors for the boxes. Give each box a suitable name
  and add a label for the y-axis as well as the main title. Do you see a
  difference in systolic blodd pressure between diabetics and
  non-diabetics?\\
\item
  Plot the variable \texttt{bmi} against \texttt{educ}. Interpret the
  picture you see.\\
\item
  Plot the histogram of the high-density lipoprotein (HDL) cholesterol
  levels. How does the distribution look like? Can you convert the
  variable \texttt{hdl} so that its distribution looks more normal?
  Create such variable and add it to your data set. Save the new data
  set in a new file.
\end{itemize}

\end{document}
