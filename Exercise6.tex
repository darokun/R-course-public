\documentclass[]{article}
\usepackage[T1]{fontenc}
\usepackage{lmodern}
\usepackage{amssymb,amsmath}
\usepackage{ifxetex,ifluatex}
\usepackage{fixltx2e} % provides \textsubscript
% use upquote if available, for straight quotes in verbatim environments
\IfFileExists{upquote.sty}{\usepackage{upquote}}{}
\ifnum 0\ifxetex 1\fi\ifluatex 1\fi=0 % if pdftex
  \usepackage[utf8]{inputenc}
\else % if luatex or xelatex
  \ifxetex
    \usepackage{mathspec}
    \usepackage{xltxtra,xunicode}
  \else
    \usepackage{fontspec}
  \fi
  \defaultfontfeatures{Mapping=tex-text,Scale=MatchLowercase}
  \newcommand{\euro}{€}
\fi
% use microtype if available
\IfFileExists{microtype.sty}{\usepackage{microtype}}{}
\usepackage[tmargin=1.5in]{geometry}
\ifxetex
  \usepackage[setpagesize=false, % page size defined by xetex
              unicode=false, % unicode breaks when used with xetex
              xetex]{hyperref}
\else
  \usepackage[unicode=true]{hyperref}
\fi
\hypersetup{breaklinks=true,
            bookmarks=true,
            pdfauthor={Riccardo De Bin and Vindi Jurinovic},
            pdftitle={Exercise 7},
            colorlinks=true,
            citecolor=blue,
            urlcolor=blue,
            linkcolor=magenta,
            pdfborder={0 0 0}}
\urlstyle{same}  % don't use monospace font for urls
\setlength{\parindent}{0pt}
\setlength{\parskip}{6pt plus 2pt minus 1pt}
\setlength{\emergencystretch}{3em}  % prevent overfull lines
\setcounter{secnumdepth}{5}

%%% Change title format to be more compact
\usepackage{titling}
\setlength{\droptitle}{-2em}
  \title{Exercise 7}
  \pretitle{\vspace{\droptitle}\centering\huge}
  \posttitle{\par}
  \author{Riccardo De Bin and Vindi Jurinovic}
  \preauthor{\centering\large\emph}
  \postauthor{\par}
  \date{}
  \predate{}\postdate{}




\begin{document}

\maketitle


\textbf{Exercise 1:}

\begin{enumerate}
\def\labelenumi{\arabic{enumi}.}
\item
  Categorize the variable \texttt{bmi} into an underweight
  ($BMI < 18.5$), normal weight ($18.5 \leq BMI < 25$), overweight
  ($25 \leq BMI < 30$) and obese ($BMI \geq 30$) group. Turn the
  variable into a factor. What is the proportion of overweight or obese
  people according to the categorized BMI? What is the proportion of
  people ever diagnosed with being overweight (variable
  \texttt{ovrwght\_ever})? How many overweight people were actually ever
  diagnosed with being overweight?
\item
  Is there a difference in diabetes prevalence between obese people
  diagnosed with overweight and those who were never diagnosed? What
  about self-rated health? How do you explain the results?
\item
  With a function \texttt{sample}, you can create a random subsample of
  a data set. For example, the command \texttt{sample(1:1000, 500)}
  creates a subsample of size 500 from a vector containing numbers
  between 1 and 1000. Create a subsample of size 500 using your data
  set. In this subsample, test the relationship between heart diseases
  and gender. Repeat the analysis 10 times (with 10 different
  subsamples) and note the results. Compare them with the result for the
  whole data set. How do you explain the differences?
\item
  We have seen in a previous lecture that in our data set, smoking seems
  to be negatively associated with cancer: cancer prevalence in
  non-smokers is higher than in smokers. Further analysis revealed age
  as a possible confounder: older people get cancer more often, and
  smokers tend to be younger. When we adjust for age, we can see that
  young smokers actually do tend to develop more cancers than young
  non-smokers.\\ For the following exercise, create a binary age
  (younger: age $\leq$ 50, and older: age\textgreater{}50) and a binary
  smoking variable.
\end{enumerate}

\begin{quote}
You are working for a tobacco company and want to show that smoking is
good for you regardless of your age. Your employer provides you with our
data set and asks you to produce a significant association between
smoking and cancer prevalence in every age group (of course, a desirable
association). Your task is to produce a subsample where in the end,
smoking turns out to be protective against cancer, even when you adjust
for age. Can you satisfy your employer and create such data set?
\end{quote}

\begin{quote}
Having in mind this last exercise, do you trust every result with a
significant p-value?
\end{quote}

\end{document}
