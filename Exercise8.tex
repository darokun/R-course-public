\documentclass[]{article}
\usepackage[T1]{fontenc}
\usepackage{lmodern}
\usepackage{amssymb,amsmath}
\usepackage{ifxetex,ifluatex}
\usepackage{fixltx2e} % provides \textsubscript
% use upquote if available, for straight quotes in verbatim environments
\IfFileExists{upquote.sty}{\usepackage{upquote}}{}
\ifnum 0\ifxetex 1\fi\ifluatex 1\fi=0 % if pdftex
  \usepackage[utf8]{inputenc}
\else % if luatex or xelatex
  \ifxetex
    \usepackage{mathspec}
    \usepackage{xltxtra,xunicode}
  \else
    \usepackage{fontspec}
  \fi
  \defaultfontfeatures{Mapping=tex-text,Scale=MatchLowercase}
  \newcommand{\euro}{€}
\fi
% use microtype if available
\IfFileExists{microtype.sty}{\usepackage{microtype}}{}
\usepackage[tmargin=1.5in]{geometry}
\ifxetex
  \usepackage[setpagesize=false, % page size defined by xetex
              unicode=false, % unicode breaks when used with xetex
              xetex]{hyperref}
\else
  \usepackage[unicode=true]{hyperref}
\fi
\hypersetup{breaklinks=true,
            bookmarks=true,
            pdfauthor={Riccardo De Bin and Vindi Jurinovic},
            pdftitle={Exercise 8},
            colorlinks=true,
            citecolor=blue,
            urlcolor=blue,
            linkcolor=magenta,
            pdfborder={0 0 0}}
\urlstyle{same}  % don't use monospace font for urls
\setlength{\parindent}{0pt}
\setlength{\parskip}{6pt plus 2pt minus 1pt}
\setlength{\emergencystretch}{3em}  % prevent overfull lines
\setcounter{secnumdepth}{5}

%%% Change title format to be more compact
\usepackage{titling}
\setlength{\droptitle}{-2em}
  \title{Exercise 8}
  \pretitle{\vspace{\droptitle}\centering\huge}
  \posttitle{\par}
  \author{Riccardo De Bin and Vindi Jurinovic}
  \preauthor{\centering\large\emph}
  \postauthor{\par}
  \date{}
  \predate{}\postdate{}




\begin{document}

\maketitle


\textbf{Exercise 7:} (Data Project)

\begin{enumerate}
\def\labelenumi{\alph{enumi})}
\item
  How strong is the relationship between BMI and systolic blood
  pressure? Is it significant? How much of the variation in systolic
  blood pressure can be explained (in a statistical sense) by variation
  in BMI?
\item
  Does the relationship between systolic blood pressure and BMI change
  when you adjust for age (categorized)? Interpret the coefficients of
  the resulting model (when you mean-center BMI before fitting the
  model, you can also interpret the intercept). Would you say that BMI
  has a clinically relevant impact on blood pressure, according to your
  model?
\item
  Try to find a better model to predict systolic blood pressure by
  including more covariates. Select a number of candidate covariates
  which in your opinion may be related to systolic blood pressure, and
  then choose a model selection strategy and a criterion/test for
  comparing models. Describe the model with the best fit according to
  your search, and interpret the model coefficients.
\end{enumerate}

\textbf{Exercise 8:} (Data Project)

\begin{enumerate}
\def\labelenumi{\alph{enumi})}
\item
  Analyze the relationship between lifetime diagnosis of cancer and
  exposure to pollutants, using the categorized age variable (Note: No
  information on pollutant exposure was collected from participants aged
  80+, so these cannot be included in the analysis). Does the adjustment
  for age change the picture? Interpret the model coefficients including
  the intercept.
\item
  Try to find a good model of cancer diagnosis. Describe and interpret
  it as you did for systolic blood pressure.
\end{enumerate}

\textbf{Exercise 9:} (Data Project)

~~~~~Dichotomize the variable \texttt{sweet\_prvmo} into the categories
`\textless{} 30 portions' and `$\geq$ 30 portions'. Dichotomize
\texttt{diab\_lft} into `No Diabetes' vs. `Diabetes or Prediabetes'.
Test the relationship between the two dichotomous variables. Before you
look at the results: Do you think there is an association, and if so, in
which direction? Why do you think so? Compare the results with your
guess.

\end{document}
