\documentclass[]{article}
\usepackage[T1]{fontenc}
\usepackage{lmodern}
\usepackage{amssymb,amsmath}
\usepackage{ifxetex,ifluatex}
\usepackage{fixltx2e} % provides \textsubscript
% use upquote if available, for straight quotes in verbatim environments
\IfFileExists{upquote.sty}{\usepackage{upquote}}{}
\ifnum 0\ifxetex 1\fi\ifluatex 1\fi=0 % if pdftex
  \usepackage[utf8]{inputenc}
\else % if luatex or xelatex
  \ifxetex
    \usepackage{mathspec}
    \usepackage{xltxtra,xunicode}
  \else
    \usepackage{fontspec}
  \fi
  \defaultfontfeatures{Mapping=tex-text,Scale=MatchLowercase}
  \newcommand{\euro}{€}
\fi
% use microtype if available
\IfFileExists{microtype.sty}{\usepackage{microtype}}{}
\usepackage[tmargin=1.5in]{geometry}
\ifxetex
  \usepackage[setpagesize=false, % page size defined by xetex
              unicode=false, % unicode breaks when used with xetex
              xetex]{hyperref}
\else
  \usepackage[unicode=true]{hyperref}
\fi
\hypersetup{breaklinks=true,
            bookmarks=true,
            pdfauthor={Riccardo De Bin and Vindi Jurinovic},
            pdftitle={Exercise 3},
            colorlinks=true,
            citecolor=blue,
            urlcolor=blue,
            linkcolor=magenta,
            pdfborder={0 0 0}}
\urlstyle{same}  % don't use monospace font for urls
\setlength{\parindent}{0pt}
\setlength{\parskip}{6pt plus 2pt minus 1pt}
\setlength{\emergencystretch}{3em}  % prevent overfull lines
\setcounter{secnumdepth}{5}

%%% Change title format to be more compact
\usepackage{titling}
\setlength{\droptitle}{-2em}
  \title{Exercise 3}
  \pretitle{\vspace{\droptitle}\centering\huge}
  \posttitle{\par}
  \author{Riccardo De Bin and Vindi Jurinovic}
  \preauthor{\centering\large\emph}
  \postauthor{\par}
  \predate{\centering\large\emph}
  \postdate{\par}
  \date{November 04, 2015}




\begin{document}

\maketitle


Today we have seen how to perform a t-test, an F-test and a proportion
test, and that the R-output of these tests provides the confidence
interval for the test statistic. We want to practice more with these
tools.

\begin{itemize}
\itemsep1pt\parskip0pt\parsep0pt
\item
  If we want to test if the variance in height is the same in women and
  men, which test can we perform? Can we reject the null hypothesis of
  equal variances?\\
\item
  Which test can we use if, instead, we want to check if males are on
  average taller than females? Set an adequate alternative hypothesis.\\
\item
  Analyze the confidence interval obtained in the previous point. Why
  doesn't it have an upper bound?\\\\
\item
  Now look at the distribution of the variable \texttt{weight}: can we
  graphically state its normality? Perform a transformation in order to
  recover it.\\
\item
  Test if the mean of the variable \texttt{weight} is 80 kg, testing
  $H_0 : log(weight) = log(80)$ versus
  $H_1 : log(weight) \neq log(80)$.\\
\item
  Use the previous command to give an interval estimate for the weight,
  in particular a confidence interval with level 0.99.\\\\
\item
  Provide a punctual and an interval estimate for the prevalence of
  heart diseases and lung pathology.\\
\item
  Test if the prevalence is statistically different between men and
  women.
\end{itemize}

\end{document}
